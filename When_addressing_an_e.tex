When addressing an element by name, you can reduce typing by hitting the TAB-key after entering \lstinline{blobInfo.} --- this will display all the field-names in the structure.

It is important to realize that \lstinline{imread} will behave different for different image formats.
For example, the tiff format used here supports the reading of specific images from a stack via the \lstinline{'index'} input argument (illustrated below) and extraction of pixel regions via the \lstinline{'pixelregion'} input argument. The latter is very useful when images are large or many as it can speed up processing not having to read the entire image image into memory.
On the other hand, jpeg2000 supports \lstinline{'pixelregion'} and \lstinline{'reductionlevel'}, but not \lstinline{'index'}.

\subsection{Reading and displaying an image-stack}
Taking one step up in complexity we will now work with a stack of tiff-files instead.
These are the steps we will go through

\begin{enumerate}
\item Open ``MRI Stack (528K)'' in ImageJ (File > Open Samples)
\item Save the stack to your desktop, or some other place where you can find it (File > Save)
\item Load a single image from the stack into a two-dimensional variable
\item Load a multiple images from the stack into a three-dimensional variable
\item Browse through the stack using the \lstinline{implay} command
\item Create a montage of all the images using the \lstinline{montage} command
\end{enumerate}
After performing the first two steps in ImageJ, we switch to MATLAB: