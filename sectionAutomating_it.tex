\section{Automating it}
The command-line is a wonderful place to quickly try out new ideas --- just type it in and hit return.
Once these ideas become longer we need to somehow record them in one place so that we can repeat them later on without having to type everything again.
You know what we are getting to: The creation of computer programs.

In the simplest of cases we can take a series of commands, that were executed in the command line, and save them to a file. We could then, at a later stage, open that file and copy these lines into the command line, one after the other, and press return.  This is actually a pretty accurate description of what takes place when MATLAB runs a script: It goes through each line of the script and tries to execute it, one after the other, starting at the top of the file.

\subsection{Create, save, and run scripts}
You can use any editor you want for writing down your collection of MATLAB statements.
For ease of use, proximity, uniformity, and because it comes with many powerful extra features, we shall use the editor that comes with MATLAB\@.
It will look something like in Fig.~\ref{fig:theEditor} for a properly typeset and documented program. 
You will recognize most of the commands from when we plotted the sinusoidsal functions earlier.
But now we have also added some text to explain what we are doing.