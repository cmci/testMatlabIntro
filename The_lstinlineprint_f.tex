The \lstinline{print} function is not confined to the pdf format but can also export to png, tiff, jpeg, etc.

\subsection{Make it pretty}
We have a large degree of control over how things are rendered in MATLAB\@. 
It is possible to set the typeface, font, colors, line-thickness, plot symbols, etc.
Don't overdo this! The main objective is to communicate your message, and that message is rarely ``look how many colors I have'' --- if you only have two graphs in the same figure, gray-scale will likely suffice. Strive for clarity!

\subsection{Getting help}
At this point you might want to know how to get help for a specific command.
That is easy, simply type \lstinline{help} and then the name of the command you need help on.
Example, for the \lstinline{xlabel} command we just used: